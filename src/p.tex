% !TeX root =./x2.tex
% !TeX program = pdfpLaTeX

\LaTeX は, 正しい使い方をすれば,
それなりの見栄えの文書ができるシステムです.
見栄えがおかしいと感じるときは,
誤った使い方をしているとき,
もしくは,
そもそも文書作成の常識を知らないためその様に思うだけ
という場合が少なくありません.
これらは,
知らなければどうにもならないものですのですが,
一度知ってしまえばなんてことはないものなので.
ここにまとめておこうと思います.

\chapter{基本的な注意}
\section{基本的な方針や技術的なことなどについて}
\LaTeX
は,
ソースと呼ばれるテキストファイルを編集し,
それをPDFなどに変換する (コンパイルする)
ことによって文書を作成するシステムです.

ソースをPDFなどに
変換するソフトウエア (コンパイラ) が
\LaTeX
です.
歴史的な事情はさておき,
現在は色々と過渡期にあり,
コンパイルに使われる
\LaTeX
と呼ばれるソフトウエアには
いくつかのバリアントがある状況です.
特に日本語が通るものについては,
事情が複雑な状況です.
TeXWorksなどのエディタでは,
これらを選択できる様になっていると思います.

\LaTeX
文書を作成するにあたって
`モード' と呼ばれる概念があります.
大雑把には,
通常の文章を書くためのモード,
数式を書くための数式モード.

\section{ソースの改行について}
ソースファイルにおける改行は,
基本的にはPDFになったときに改行として現れることはありません.
したがって,
ソースファイルでは,
文の途中で改行をしても問題ありません.
こまめに改行をするようにしましょう.
例えば, 句読点があったところなどのような
文の途中で改行を入れましょう.
このように, こまめに改行をすると,
ソースの何行目と言ったときに指し示す範囲が狭くなるので,
エラーの解消や,
改訂の指示などのときに,
非常に便利です.

たとえば, 以下の様に書くと, 10行目でエラーがでたとしても,
どこにその原因があるのかすぐにはわかりません.

一方, 以下の様に書くと, 10行目でエラーがあるといわれれば,
エラーの原因を特定するのは簡単です.


\section{図について}
最終的には, Tikzなりなんなりを使い
十分な解像度をもってきれいに描かれた図を入れることになります.
しかし, 図を実際に描くのは時間がかかる工程なので,
最初の段階では図自体を入れる必要はありません.
図を入れたほうが良い場所には,
図を入れるスペースだけを用意する程度で十分で,
それ以上のことをする必要はありません.
今後, 改訂が進むうちに結局その図を使わないなどということも起こります.
また, 紙面の都合で内容を精査する必要が出てきたときに,
`せっかく時間をかけたから' などというナンセンスな理由で,
削除しにくくなったりし, 不毛です.
図は載せることが確定してから描くというので十分です.

もしスペースを入れるだけではどんな図を入れようと思っていたか忘れそうな場合は,
手書きで紙に図を描き
それを写真にとったものを
一時的に載せておけば良いです.

いずれにせよ, 初期段階では図に時間をかけてはいけません.
本文に注力すべきです.

\section{句読点について}

数式漢字かな混じりの技術的文書を書く場合,
句読点は
マル`\verb|。|'とテン`\verb|、|'
ではなく,
ピリオド`\verb|.|'とコンマ`\verb|,|'
を用いるのが無難です.%
\footnote{マルとテンを使うよう特段の指示があるならそれに従うべきです}\@
また, 英語における句読法を準用し,
コロン`\verb|:|'とセミコロン`\verb|;|'も用いると読みやすくなることが多いです.


句読点としてピリオドとカンマを用いる場合,
全角の`\verb|.|'と`\verb|,|'と
半角の`\verb|.|'と`\verb|,|'がありますが,
半角の`\verb|.|'と`\verb|,|'に統一する方が無難です.%
\footnote{全角を使うよう特段の指示があるならそれに従うべきです}\@
少なくとも数式中に現れるピリオドやコンマは半角を使うべきです.
文章中は全角, 数式中は半角と使い分けるのは大変ですので,
半角で統一するのが無難です.

(数式モードではない部分について)
半角の`\verb|.|'と`\verb|,|'を用いる場合,
句読点のあとには, 半角のスペースを入れないといけません.
半角のスペースは1つ以上入っていれば,
いくつ入っていても問題ありません.
スペースを入れず
  `\verb|hoge,fuga.piyo|'
のように書くと,
\begin{quotation}hoge,fuga.piyo\end{quotation}
のようになり不自然につまります.
適切に,
スペースを入れて
  `\verb|hoge, fuga. piyo|'
のように書くと,
\begin{quotation}hoge, fuga. piyo\end{quotation}
  のようになります.
半角のコロンやセミコロンも同様にスペースを入れて使います.


半角のピリオドやカンマに半角スペースを入れて正しく使えば,
適切にスペーシングをしてくれます.
例えば,
ピリオドのあとは文が切れたスペースとして扱われ,
カンマのあとは単語間のスペースとして扱われます.
例えば, 文が切れたあとのスペース
の方が単語間のスペースよりも若干広くなるなどし,
読みやすくなることがあります.


\begin{remark}
  数式モードではない部分では,
  ピリオドのあとのスペースは,
  原則として, 文が切れたあとのスペースとして扱われます.

  英語では,
  ピリオドは句点としても扱われますが,
  省略を表す場合にも用いられます.
  省略を表すピリオドが文中で用いられる場合には,
  そこでは文は終わらないので,
  そのあとのスペースは単語間のスペースであるべきです.
  \LaTeX
  では,
  大文字の直後にピリオドが来ている場合,
  例外的に,
  単語間のスペースとして認識されます.
  たとえば,
  `\verb|A. Inoki|'
  のように書けば適切に単語感のスペースとなります.
  したがって, 特に何もせずともうまく行きます.

  意図的に単語間のスペースを入れる場合には,
  `\verb|\ |'
  を使います.
  また, 意図的に文の区切りのスペースを入れる場合には,
  `\verb|\@|'
  を使います.
  これらを使う場面はほとんどありません.
  例えば,
  `Mr.\ X'
  のように,
  小文字の直後のピリオドが略語を表すピリオドだった場合,
  `\verb|Mr.\ X|'
  のように, 
  `\verb|\ |'
  を使います.
  また,
  例えば,
  `He is Mr.\ X.\@'
  のように,
  文の最後が大文字で終わる場合,
  略語ではなく文の区切りであることを明示し
  `\verb|He is Mr.\ X.\@|'
  のように
  `\verb|\@|'
  を使って書きます.
  
\end{remark}


数式では,
例えば,
カンマはベクトルを書くときの数の区切り,
コロンは比を表す数の区切り,
などとして扱われ,
文を区切る句読点ではないことがしばしばあります,
  \LaTeX
  では,
数式モードとそれ以外では,
ピリオドやカンマなどの扱いが異なります.
数式モードでは,
句読点ではなく`区切り文字'として扱います.
ベクトルや比を表す場合に前後で行が変わると読みにくくなるので,
改行が抑制され詰め気味に組まれます.
直後に半角スペースを入れても組み方は変わりません.

句読点のピリオドやカンマは
数式モードに入れてはいけません.
たとえば, `$x=1$, $y=1$'
のように列挙する場合は,
\verb|$x=1$, $y=1$|
のようにカンマを数式モードの外に出し,
句読点として扱った方が良いです.
\verb|$x=1, y=1$|
のように書くと, \verb|,|で改行をするのを避けようとするため,
不自然なところで改行されてしまうことがしばしばあります.

一方ベクトル$(1,2,3)$を表すなら,
\verb|$(1,2,3)$|
とか
\verb|$(1, 2, 3)$|
と書くべきで,
\verb|($1$, $2$, $3$)|
は望ましくありません.

\begin{remark}
数式モードでは, $:$の前後には均等にスペースが入ります.
比を表すような場合$1:2$のように均等にスペースが入るので,
問題ありません.
一方. 写像を表す際にコロンを使う習慣があります.
例えば, $X$から$Y$への写像$\varphi$を表すのに,
$\varphi:X\to Y$のように書くことがあります.
この表記を書く際に, 均等な配置の
$\varphi:X\to Y$を書く人もいます\footnote{おそらく多くの人が均等な配置のコロンを使っているように思います}が,
$\varphi\colon X\to Y$のように非対称なスペーシングが
望ましいと考える人もいます.
このような非対称なスペーシング行われるコロンとして,
`\verb|\colon|'という数式モード用のコマンドがあります.
`\verb|$\varphi\colon X\to Y$|'の様に使います.
写像のコロンに関しては,
どちらでも構わないと思いますが,
一つの文書の中では統一されるべきだと思います.
\end{remark}


\section{参考文献について}


\section{段落について}
段落を新しくするためには,
空行を入れます.
改行を\lstinline{\\}でいれるなどと言った真似をしてはいけません.
また, 段落開始時にインデントをするのに,
全角のスペースや\lstinline{\quad}などを入れてはいけません.
インデントをどうするかなどは,
スタイルで指定するべきことです.
まずは内容に集中しましょう.

\begin{remark}
段落を指定する
\lstinline{\par}というコマンドは, 空行で代用できます.
(1行以上の空行は, \lstinline{\par}に置き換えられます).
コマンドよりも空行のほうがソースの可読性が上がるので
\lstinline{\par}は用いず空行を使います.
\end{remark}

\section{数式モードの適切な使用について}

\section{定理環境について}
定理などは, theorem環境などを定義しそれを使います.
まずは, 以下をプリアンブルに入れます.
\begin{lstlisting}
\usepackage{amsthm}

\begin{lstlisting}
\theoremstyle{definition} 
% section番号を入れる場合
%\newtheorem{thm}{Theorem}[section]
% 入れない場合
\newtheorem{thm}{Theorem}

\newtheorem{theorem}[thm]{定理}
\newtheorem{lemma}[thm]{補題}
\newtheorem{definition}[thm]{定義}
\end{lstlisting}

その上で,
定義なら
\begin{lstlisting}
\begin{definition}
....
\end{definition}
\end{lstlisting}
のように書きます.
定義なら
\lstinline{theorem},
補題なら
\lstinline{lemma}
をつかます.

定理にタイトルを入れたい場合は
\begin{lstlisting}
\begin{theorem}[零点定理]
...
\end{theorem}
\end{lstlisting}
のように \lstinline{[]} で書きます.

また, 証明は,
\begin{lstlisting}
\begin{proof}
...
\end{proof}
\end{lstlisting}
のようにproof環境を使います.

\section{定義される用語の強調について}
通常の数学書では,
定義を書く際, 定義される用語を,
イタリック(もしくはボールド)で強調する監修があります.
定義される用語は
\lstinline{\emph}
で強調をしましょう.
\begin{lstlisting}
The pair is called a \emph{directed graph}.
\end{lstlisting}
のように使います.


\begin{remark}
強調のされ方はスタイルによって,
イタリックであったりボールドであったりします.
それらは, スタイルで指定することであって,
内容をまとめるときに気にするべきことではありません.
イタリックにするとかボールドにするとかではなく,
強調を指示する
\lstinline{\emph}
を使います.
\end{remark}

\section{別組の数式モードについて/ナンバリング/インラインセンタリング}

\section{スペースファクターについて}
(数式ではない)本文中において,
\lstinline{\\}による改行や
\lstinline{\quad},
\lstinline{\hspace},
\lstinline{\hskip}などによる水平方向の空白,
\lstinline{\vspace},
\lstinline{\vskip}などによる垂直方向の空白は使いません.

新しく段落を始めるときには,
空行を入れます.
数式を新しい行から始める場合は,
別組の数式を使います.
いずれにせよ, \lstinline{\\}などは使いません.

\begin{remark}
  本文中で使う可能性があるのは, \lstinline{\ }と\lstinline{\@}のみです.
  これも非常に特殊な場合に使うものであって,
  まず, ほとんどの場合に使うことはありません.
  本文中でコマンドでスペースを挿入することはまずないと考えて良いと思います.

  もしかしたら, 段落を変えたあとインデントをしたいなどと思うことが有るかもしれませんが, それらはスタイルで指定するべきものであって,
  スペースをいちいち挿入して調整するものではありません.
  内容に集中しましょう.
\end{remark}


\section{箇条書きについて}

箇条書きは基本的には手書きをせず,
コマンドを使って書きます.
番号をふる場合は
\begin{lstlisting}
\begin{enumerate}
\item .....
\item ......
\end{enumerate}
\end{lstlisting}
のようにします.
番号を振らず点を打つだけなら,
\begin{lstlisting}
\begin{itemize}
\item ....
\item ......
\end{itemize}
\end{lstlisting}
のようにします.

見出し付きの箇条書きは,
\begin{lstlisting}
\begin{description}
\item[Ax1] ....
\item[Ax2] ......
\end{description}
\end{lstlisting}
のようにします.
\begin{remark}
  箇条書きをするに当たって,
  番号の振り方をローマ数字にしたいとか,
  片側のみのカッコにしたいと,
  思うことが有るかもしれません.
それらは, スタイルで指定することであって,
内容をまとめるときに気にするべきことではありません.
そういう部分は気にせず,
内容に集中しましょう.
\end{remark}

\section{別組数式の数式番号について}
他で参照しないなら
数式に番号は振りません.
\lstinline{align}
環境などではなく
\lstinline{align*}
を使いましょう.


\section{レイアウトについて}


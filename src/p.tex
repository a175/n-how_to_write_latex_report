% !TeX root =./x2.tex
% !TeX program = pdfpLaTeX



\chapter{技術的な注意}

\section{句読点について}

数式漢字かな混じりの技術的文書を書く場合,
句読点は
マル`\verb|。|'とテン`\verb|、|'
ではなく,
ピリオド`\verb|.|'とコンマ`\verb|,|'
を用いるのが無難です.%
\footnote{マルとテンを使うよう特段の指示があるならそれに従うべきです}\@
また, 英語における句読法を準用し,
コロン`\verb|:|'とセミコロン`\verb|;|'も用いると読みやすくなることが多いです.


句読点としてピリオドとカンマを用いる場合,
全角の`\verb|.|'と`\verb|,|'と
半角の`\verb|.|'と`\verb|,|'がありますが,
半角の`\verb|.|'と`\verb|,|'に統一する方が無難です.%
\footnote{全角を使うよう特段の指示があるならそれに従うべきです}\@
少なくとも数式中に現れるピリオドやコンマは半角を使うべきです.
文章中は全角, 数式中は半角と使い分けるのは大変ですので,
半角で統一するのが無難です.

(数式モードではない部分について)
半角の`\verb|.|'と`\verb|,|'を用いる場合,
句読点のあとには, 半角のスペースを入れないといけません.
半角のスペースは1つ以上入っていれば,
いくつ入っていても問題ありません.
スペースを入れず
  `\verb|hoge,fuga.piyo|'
のように書くと,
\begin{quotation}hoge,fuga.piyo\end{quotation}
のようになり不自然につまります.
適切に,
スペースを入れて
  `\verb|hoge, fuga. piyo|'
のように書くと,
\begin{quotation}hoge, fuga. piyo\end{quotation}
  のようになります.
半角のコロンやセミコロンも同様にスペースを入れて使います.


半角のピリオドやカンマに半角スペースを入れて正しく使えば,
適切にスペーシングをしてくれます.
例えば,
ピリオドのあとは文が切れたスペースとして扱われ,
カンマのあとは単語間のスペースとして扱われます.
例えば, 文が切れたあとのスペース
の方が単語間のスペースよりも若干広くなるなどし,
読みやすくなることがあります.


\begin{remark}
  数式モードではない部分では,
  ピリオドのあとのスペースは,
  原則として, 文が切れたあとのスペースとして扱われます.

  英語では,
  ピリオドは句点としても扱われますが,
  省略を表す場合にも用いられます.
  省略を表すピリオドが文中で用いられる場合には,
  そこでは文は終わらないので,
  そのあとのスペースは単語間のスペースであるべきです.
  \LaTeX
  では,
  大文字の直後にピリオドが来ている場合,
  例外的に,
  単語間のスペースとして認識されます.
  たとえば,
  `\verb|A. Inoki|'
  のように書けば適切に単語感のスペースとなります.
  したがって, 特に何もせずともうまく行きます.

  意図的に単語間のスペースを入れる場合には,
  `\verb|\ |'
  を使います.
  また, 意図的に文の区切りのスペースを入れる場合には,
  `\verb|\@|'
  を使います.
  これらを使う場面はほとんどありません.
  例えば,
  `Mr.\ X'
  のように,
  小文字の直後のピリオドが略語を表すピリオドだった場合,
  `\verb|Mr.\ X|'
  のように, 
  `\verb|\ |'
  を使います.
  また,
  例えば,
  `He is Mr.\ X.\@'
  のように,
  文の最後が大文字で終わる場合,
  略語ではなく文の区切りであることを明示し
  `\verb|He is Mr.\ X.\@|'
  のように
  `\verb|\@|'
  を使って書きます.
  
\end{remark}


数式では,
例えば,
カンマはベクトルを書くときの数の区切り,
コロンは比を表す数の区切り,
などとして扱われ,
文を区切る句読点ではないことがしばしばあります,
  \LaTeX
  では,
数式モードとそれ以外では,
ピリオドやカンマなどの扱いが異なります.
数式モードでは,
句読点ではなく`区切り文字'として扱います.
ベクトルや比を表す場合に前後で行が変わると読みにくくなるので,
改行が抑制され詰め気味に組まれます.
直後に半角スペースを入れても組み方は変わりません.

句読点のピリオドやカンマは
数式モードに入れてはいけません.
たとえば, `$x=1$, $y=1$'
のように列挙する場合は,
\verb|$x=1$, $y=1$|
のようにカンマを数式モードの外に出し,
句読点として扱った方が良いです.
\verb|$x=1, y=1$|
のように書くと, \verb|,|で改行をするのを避けようとするため,
不自然なところで改行されてしまうことがしばしばあります.

一方ベクトル$(1,2,3)$を表すなら,
\verb|$(1,2,3)$|
とか
\verb|$(1, 2, 3)$|
と書くべきで,
\verb|($1$, $2$, $3$)|
は望ましくありません.

\begin{remark}
数式モードでは, $:$の前後には均等にスペースが入ります.
比を表すような場合$1:2$のように均等にスペースが入るので,
問題ありません.
一方. 写像を表す際にコロンを使う習慣があります.
例えば, $X$から$Y$への写像$\varphi$を表すのに,
$\varphi:X\to Y$のように書くことがあります.
この表記を書く際に, 均等な配置の
$\varphi:X\to Y$を書く人もいます\footnote{おそらく多くの人が均等な配置のコロンを使っているように思います}が,
$\varphi\colon X\to Y$のように非対称なスペーシングが
望ましいと考える人もいます.
このような非対称なスペーシング行われるコロンとして,
`\verb|\colon|'という数式モード用のコマンドがあります.
`\verb|$\varphi\colon X\to Y$|'の様に使います.
写像のコロンに関しては,
どちらでも構わないと思いますが,
一つの文書の中では統一されるべきだと思います.
\end{remark}


\section{参考文献について}

\section{ソースの改行について}

\section{段落について}

\section{数式モードの適切な使用について}

\section{定理環境について}

\section{定義される用語の強調について}

\section{別組の数式モードについて/ナンバリング/インラインセンタリング}

\section{スペースファクターについて}

\section{箇条書きについて}

\section{レイアウトについて}

